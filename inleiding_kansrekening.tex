%        File: wat_is_kans.tex
%     Created: Sat Nov 24 02:00 PM 2012 C
% Last Change: Sat Nov 24 02:00 PM 2012 C
%
\documentclass[a4paper]{report}
\usepackage{amssymb,amsmath,amsthm,enumerate}
\usepackage[dutch]{babel}

\title{Inleiding Kansrekening \\ \small{Opgebouwd vanuit de intu\"itieve verzamelingenleer en de analyse}}
\author{Ch. B. ten Brinke}

\newtheorem{theorem}{Stelling}
\newtheorem{lemma}[theorem]{Lemma}
\newtheorem{corollary}[theorem]{Gevolg}
\newtheorem{proposition}[theorem]{Propositie}

\theoremstyle{definition}
\newtheorem{definition}[theorem]{Definitie}
\newtheorem{example}[theorem]{Voorbeeld}
\newtheorem{remark}[theorem]{Opmerking}


\begin{document}
\maketitle

\chapter{Algemene Kansruimtes}
Kansrekening is een directe toepassing van verzamelingenleer en analyse.
Het fundamentele concept in de kansrekening is de zogenaamde \emph{kansruimte}.

\section{Definitie Kansruimte}
Een kansruimte kunnen we informeel beschouwen als een verzameling, waarbij we veronderstellen dat er precies \'e\'en element van deze verzameling `waar is', zonder dat we weten welk element dat is.
Op de machtsverzameling van de kansruimte is een zekere functie gedefinieerd, die we de \emph{kansverdeling} noemen.
Deze kansverdeling noteren we vaak met $P$, en wordt ook wel kansdistributie genoemd.
Een kansverdeling voegt aan elke deelverzameling van de betreffende kansruimte de kans toe dat het element dat `waar is' in die deelverzameling zit.
In de praktijk hoeft een kansverdeling niet altijd volledig bekend te zijn.

\begin{definition}
    Een \emph{kansruimte} $\Omega$ is een verzameling met een \emph{kansverdeling} $P: \mathcal P(\Omega) \to [0,1]$  die aan de volgende eigenschappen voldoet.
    \begin{enumerate}[i]
        \item Voor elke twee disjuncte $A,B \subset \Omega$ geldt $P(A \cup B) = P(A) + P(B)$.
        \item $P(\Omega) = 1$.
    \end{enumerate}
\end{definition}

Een uitdrukking van de vorm $P(X)$ spreken we uit als ``de kans op $X$''.
Een deelverzameling van een kansruimte is opnieuw een kansruimte, als we de kansverdeling correct induceren.
\begin{proposition}
    Laat $\Omega$ een kansruimte met kansverdeling $P$ en $A \subset \Omega$ een deelverzameling.
    Dan is $A$ een kansruimte met kansverdeling $P_A(X) = \frac{P(X)}{P(A)}$.
\end{proposition}
\begin{proof}
    We moeten laten zien dat $P_A$ aan de definitie van de kansverdeling voldoet.
    Laat $X,Y \subset A$ twee disjuncte verzamelingen.
    Er geldt nu ook dat $X,Y \subset \Omega$.
    Dus \[ P_A(X \cup Y) = \frac{P(X \cup Y)}{P(A)} = \frac{P(X) + P(Y)}{P(A)} = \frac{P(X)}{P(A)} + \frac{P(Y)}{P(A)} = P_A(X) + P_A(Y) \]
    Dus $P_A$ is een kansverdeling en daarmee is $A$ een kansruimte.
\end{proof}

Hieronder volgen een aantal handige resultaten, die direct met de definitie te bewijzen zijn.
\begin{proposition}[Verschilregel]
    \label{verschilregel}
    Laat $\Omega$ een kansruimte met kansverdeling $P$ en $B \subset A \subset \Omega$.
    Dan $P(A - B) = P(A) - P(B)$.
\end{proposition}
\begin{proof}
    We zien
    \begin{align*}
        P(A)    &= P( (A - A \cap B) \cup (A \cap B) ) \\
                &= P(A - A \cap B ) + P(A \cap B) & \text{definitie kansverdeling}
    \end{align*}
    zodat $P(A - A \cap B ) = P(A) - P(A \cap B) $.
\end{proof}

\begin{proposition}[Somregel]
    \label{somregel}
    Laat $\Omega$ een kansruimte met kansverdeling $P$ en $A,B \subset \Omega$.
    Dan $P(A \cup B) = P(A) + P(B) - P(A \cap B)$.
\end{proposition}
\begin{proof}
    De deelruimtes $A$ en $B$ zijn niet noodzakelijkerwijs disjunct, maar we kunnen $A \cup B$ wel schrijven als de vereniging van twee disjuncte verzamelingen.
    Dus
    \begin{align*}
        P(A \cup B)     &= P( (A - A \cap B) \cup B) \\
                        &= P(A - A \cap B) + P(B)  & \text{definitie kansverdeling}\\
                        &= P(A) - P(A \cap B) + P(B) & \text{verschilregel, propositie~\ref{verschilregel}}
    \end{align*}
    hetgeen de gevraagde identiteit bewijst.
\end{proof}

\begin{proposition}[Complementregel]
    \label{complementregel}
    Laat $\Omega$ een kansruimte met kansverdeling $P$ en $A \subset \Omega$.
    Dan $P(A^c) = 1 - P(A)$.
\end{proposition}
\begin{proof}
    Er geldt $P(A^c) = P(\Omega - A)$.
    Bovendien
    \begin{align*}
        1 &= P(\Omega) & \text{definitie kansverdeling}\\
          &= P( (\Omega - A) \cup A) \\
          &= P(\Omega - A) + P(A) & \text{definitie kansverdeling}
    \end{align*}
    Dus $P(A^c) = 1 - P(A)$.
\end{proof}


\section{Voorwaardelijke kans}
Nu kunnen we de notie van voorwaardelijke kans invoeren.
Hierbij beperken we de kansverdeling tot een zekere deelruimte.
We kunnen dit interpreten alsof we weten dat het `ware element' in deze deelruimte zit.
\begin{definition}
    Laat $\Omega$ een kansruimte met kansverdeling $P$ en $A \subset \Omega$ een deelruimte.
    Definieer $P(B|A) = P_A(B) = \frac{P(A \cap B)}{P(B)}$. Dit spreken we uit als `de kans van $B$ gegeven $A$'.
\end{definition}

\begin{definition}
    Laat $\Omega$ een kansruimte met kansverdeling $P$ en $A,B \subset \Omega$ twee deelruimtes.
    Dan heten $A$ en $B$ onafhankelijk als $P(A|B) = P(A)$.
\end{definition}

\begin{proposition}
    Laat $\Omega$ een kansruimte met kansverdeling $P$ en $A,B \subset \Omega$ twee deelruimtes.
    Er geldt $P(A \cap B) = P(A)P(B)$ precies dan als $A$ en $B$ onafhankelijk zijn.
\end{proposition}
\begin{proof}
    Merk op dat de volgende vergelijkingen equivalent zijn.
    \begin{align*}
        P(A|B)                      &= P(A) \\
        \frac{P(A \cap B)}{P(B)}    &= P(A) \\
        P(A \cap B)                 &= P(A)P(B)
    \end{align*}
    Dit bewijst de gevraagde gelijkwaardigheid.
\end{proof}

Het kan ook zo zijn dat twee deelruimtes onafhankelijk zijn als we ze beperken tot een zekere derde deelruimte.
Dit noemen we voorwaardelijke onafhankelijkheid.
\begin{definition}
    Laat $\Omega$ een kansruimte met kansverdeling $P$ en $A,B,C \subset \Omega$ drie deelruimteen.
    Dan heten $A$ en $B$ onafhankelijk gegeven $C$ als $P(A \cap B|C) = P(A|C)P(B|C)$.
\end{definition}

\begin{proposition}
    Laat $\Omega$ een kansruimte met kansverdeling $P$ en $A,B,C \subset \Omega$ drie deelruimtes,
    Er geldt dat $P(A \cap B \cap C)P(C) = P(A \cap C)P(B \cap C)$ precies dan als $A$ en $B$ onafhankelijk zijn gegeven $C$.
\end{proposition}
\begin{proof}
    Merk op dat de volgende vergelijkingen equivalent zijn.
    \begin{align*}
        P(A \cap B|C)                       &= P(A|C)P(B|C) \\
        \frac{P(A \cap B \cap C)}{P(C)}     &= \frac{P(A \cap C)}{P(C)} \frac{P(B \cap C)}{P(C)} \\
        P(A \cap B \cap C)P(C)              &= P(A \cap C)P(B \cap C)
    \end{align*}
    Dit bewijst de gevraagde gelijkwaardigheid.
\end{proof}

\section{Verzamelingen van kansruimtes}
(\dots)

\section{Afbeeldingen van kansruimtes}
Gegeven een kansruimte en een afbeelding op die kansruimte, kan er een nieuwe kansruimte geconstrueerd worden.

\begin{proposition}
    Laat $X$ een kansruimte met kansverdeling $P_X$ en $g: X \to Y$ een surjectieve functie.
    Dan is $Y = g(X)$, met een kansverdeling $P_Y$ zodanig dat
    \[ P_X(A) = P_Y(g(A)) \]
    voor alle $A \subset X$, opnieuw een kansruimte.
\end{proposition}
\begin{proof}
    We moeten laten zien dat $P_Y$ aan de definitie van een kansverdeling voldoet.
    Allereerst zien we dat $P_Y(Y) = P_Y(g(X)) = P_X(X) = 1$, dus $P_Y$ voldoet aan de eerste voorwaarde.

    Merk nu op dat we elke deelruimte van $Y$ kunnen schrijven als $g(A)$ voor zekere $A \subset X$.
    Laat nu $g(A),g(B) \subset Y$ twee disjuncte verzamelingen.
    Dan zien we dat
    \begin{align*}
        P_Y(g(A) \cup g(B)) &= P_Y(g(A \cup B)) \\
                            &= P_X(A \cup B) \\
                            &= P_X(A) + P_X(B) \\
                            &= P_Y(g(A)) + P_Y(g(B))
    \end{align*}

\end{proof}


\chapter{Specifieke kansruimtes}

We zullen kansruimtes beschouwen waarbij we extra eigenschappen van de kansruimte en de bijbehorende kansverdeling veronderstellen.
Bij het toepassen van kansrekening op concrete problemen, is er vaak meer bekend over kansverdelingen, zodat kennis van specifieke veelvoorkomende kansruimtes veel gebruikt kan worden.
%Belangrijke verdelingen hebben een naam, zoals de Bernoulli verdeling en de binomiale verdeling.
%
%Bij een event A bestaande uit een opeenvolging van andere events, kunnen we A visualiseren door een boom diagram te tekenen.
%Hierdoor kunnen we er intuitiever mee redeneren, zodat we vragen omtrent A makkelijker kunnen beantwoorden.
%
%Een willekeurige variabele is een kansruimte.
%Men kan een (universele) kansruimte naar een andere kansruimte sturen via een functie.
%Als een zekere kansruimte A 5 dobbelsteenworpen beschrijft, dan kunnen we een functie f: A->N definiëren die alle ogen optelt.
%De resulterende kansruimte f(A) beschrijft dan het totaal aantal ogen.
%
%Een joint verdeling van kansruimtes A en B is het cartesisch product AxB.
%Ook hierbij kunnen we verzamelingenleer nog steeds volledig gebruiken.
%
%Als de elementen van een kansruimte uit te drukken zijn in een getal, dan kunnen we spreken over de 'verwachte waarde' van een kansruimte.
%Dit is het gemiddelde van alle elementen, gewogen door kun kans.
%Merk op dat een dergelijke kansruimte gevisualiseerd kan worden als deelverzameling van de reële as.
\end{document}


