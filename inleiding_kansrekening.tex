%        File: wat_is_kans.tex
%     Created: Sat Nov 24 02:00 PM 2012 C
% Last Change: Sat Nov 24 02:00 PM 2012 C
%
\documentclass[a4paper]{book}
\usepackage{amssymb,amsmath,amsthm,enumerate,graphicx,float,lmodern,xparse, mathtools}
\usepackage[dutch]{babel}
\usepackage{color}

\title{Inleiding Eindige Kansrekening \\ \small{Opgebouwd vanuit de intu\"{\i}tieve verzamelingenleer}}
\author{Ch.\ ten Brinke}

\newtheorem{theorem}{Stelling}
\newtheorem{lemma}[theorem]{Lemma}
\newtheorem{corollary}[theorem]{Gevolg}
\newtheorem{proposition}[theorem]{Propositie}
\newtheorem{exercise}[theorem]{Opgave}

\theoremstyle{definition}
\newtheorem{definition}[theorem]{Definitie}
\newtheorem{example}[theorem]{Voorbeeld}
\newtheorem{remark}[theorem]{Opmerking}

\NewDocumentCommand\set{mg}{%
    \ensuremath{\left\lbrace #1 \IfNoValueTF{#2}{}{\, \middle|\, #2} \right\rbrace}%
}
\NewDocumentCommand\powerset{mg}{%
    \ensuremath{\mathcal{P}(#1)}
}
\NewDocumentCommand\compl{mg}{%
    \ensuremath{\bar{#1}}
}
\newcommand{\reals}{\mathbb{R}}


%\definecolor{foo}{rgb}{0.3,0.0,0.3}
%\everymath{\color{foo}}


\begin{document}
\maketitle

\chapter{Algemene Kansruimtes}
Kansrekening is gemakkelijk op te bouwen vanuit verzamelingenleer en functies op deze verzamelingen.
We veronderstellen dat de lezer bekend is met deze concepten.

Het fundamentele begrip in de kansrekening is de zogenaamde \emph{kansruimte}.
We zullen alleen eindige kansruimtes beschouwen, omdat dit genoeg intuitie verschaft over de formele constructie van kansrekening zonder dat het al te
technisch wordt.
Bovendien zal in de praktijk, e.g.\ bij het beoefenen van statistische analyses, men altijd te maken hebben met eindige kansruimtes.
Immers, alles waar we in het dagelijks leven mee te maken hebben is eindig van aard.
Daarom duidt het woord kansruimte in het vervolg van dit document altijd een eindige kansruimte aan, en zullen oneindige kansruimtes niet centraal
staan.


\section{Kansruimte}
Een kansruimte kunnen we informeel beschouwen als een eindige verzameling van mogelijke uitkomsten van een experiment.
Hierbij veronderstellen we dat er precies \'e\'en uitkomst van deze verzameling `waar is', zonder dat we weten welke dat is.
Op deze kansruimte is een zekere functie gedefinieerd, die we de \emph{kansverdeling} noemen.
Deze kansverdeling noteren we vaak met $P$.
Een kansverdeling voegt aan elk element van de betreffende kansruimte de kans toe dat deze uitkomst waar is.
In de praktijk hoeft een kansverdeling niet altijd volledig bekend te zijn.

\begin{definition}
    Een \emph{kansruimte} $\Omega$ is een eindige verzameling met een \emph{kansverdeling} $P: \Omega \to [0,1]$  die aan de volgende eigenschappen voldoet.
    \begin{enumerate}[i]
        \item $P(\omega) \geq 0$ voor alle $\omega \in \Omega$.
        \item $\sum_{\omega \in \Omega}P(\omega) = 1$.
    \end{enumerate}

    Een uitdrukking van de vorm $P(\omega)$ spreken we uit als ``de kans op $\omega$''.
    Als $P(\omega)$ hetzelfde is voor alle $\omega\in \Omega$, dan heet $P$ een \emph{uniforme verdeling}.

    Vaak willen we de kans van een deelverzameling van $\Omega$ uitdrukken, in plaats van de kans van een individueel element.
    Daarom noteren we de kans van een deelverzameling $A \subseteq \Omega$ met $P(A) = \sum_{a \in A} P(a)$.
\end{definition}

De elementen van een kansruimte worden informeel \emph{uitkomsten} genoemd, en verzamelingen van uitkomsten worden informeel \emph{gebeurtenissen}
genoemd.

\begin{exercise}
    Laat zien dat de verzameling van mogelijke uitkomsten van het gooien van een zeszijdige dobbelsteen een kansruimte vormt met zes elementen.
\end{exercise}

Hieronder volgen een aantal handige resultaten, die direct met de definitie te bewijzen zijn.
\begin{proposition}[Somregel]\label{somregel}
    Laat $\Omega$ een kansruimte met kansverdeling $P$ en $A,B \subseteq \Omega$.
    Dan geldt $P(A \cup B) = P(A) + P(B) - P(A \cap B)$.
    In het bijzonder geldt $P(A \cup B) = P(A) + P(B)$ wanneer $A$ en $B$ disjunct zijn.
\end{proposition}
\begin{proof}
    \begin{align*}
        P(A \cup B)     &= \sum_{\omega \in A \cup B} P(\omega) & \text{definitie kansverdeling}\\
                        &= \sum_{\omega \in A} P(\omega) + \sum_{\omega \in B} P(\omega) - \sum_{\omega \in A \cap B} P(\omega) \\
                        &= P(A) + P(B) - P(A \cap B) \\
    \end{align*}
\end{proof}

\begin{proposition}[Verschilregel]\label{verschilregel}
    Laat $\Omega$ een kansruimte met kansverdeling $P$ en $B \subseteq A \subseteq \Omega$.
    Dan geldt $P(A - B) = P(A) - P(A \cap B) $.
\end{proposition}
\begin{proof}
    We zien
    \begin{align*}
        P(A)    &= P( (A - A \cap B) \cup (A \cap B) ) \\
                &= P(A - A \cap B ) + P(A \cap B) & \text{somregel} \\
                &= P(A - B) + P(A \cap B) \\
    \end{align*}
    dus $P(A - B) = P(A) - P(A \cap B) $.
\end{proof}

\begin{proposition}[Complementregel]\label{complementregel}
    Laat $\Omega$ een kansruimte met kansverdeling $P$ en $A \subseteq \Omega$.
    Noteer met $\compl{A}$ het complement van $A$, i.e.\ $\Omega - A$.
    Dan geldt $P(\compl{A}) = 1 - P(A)$.
\end{proposition}
\begin{proof}
    Er geldt $P(\compl{A}) = P(\Omega - A)$.
    Bovendien
    \begin{align*}
        1 &= P(\Omega) & \text{definitie kansverdeling}\\
          &= P( (\Omega - A) \cup A) \\
          &= P(\Omega - A) + P(A) & \text{somregel}
    \end{align*}
    Dus $P(\compl{A}) = 1 - P(A)$.
\end{proof}


\begin{exercise}
    Zij $\Omega = \set{1,2,3,4,5}$ de verzameling van uitkomsten van het gooien van een vijfzijdige dobbelsteen,
    i.e.\ $P(\set{\omega}) = \frac{1}{5}$ voor elke $\omega \in \Omega$.
    Zij $A \subseteq \Omega$ de verzameling van oneven uitkomsten.
    Bewijs dat $P(A) = \frac{3}{5}$.
    Bewijs dat $P(\compl{A}) = \frac{2}{5}$.
\end{exercise}


\section{Voorwaardelijke kans}
Nu kunnen we de notie van voorwaardelijke kans invoeren.
Hierbij beperken we de kansverdeling tot een zekere deelverzameling.
We kunnen dit interpreteren alsof we weten dat ware uitkomst in deze deelverzameling zit.
\begin{definition}
    Laat $\Omega$ een kansruimte met kansverdeling $P$ en $A \subseteq \Omega$ een deelverzameling.
    Definieer $P(B|A) = P_A(B) = \frac{P(A \cap B)}{P(B)}$. Dit spreken we uit als `de kans van $B$ gegeven $A$'.
\end{definition}

\begin{definition}
    Laat $\Omega$ een kansruimte met kansverdeling $P$ en $A,B \subseteq \Omega$ twee deelverzamelingen.
    Dan heten $A$ en $B$ onafhankelijk als $P(A|B) = P(A)$.
\end{definition}

Informeel kunnen we de onafhankelijkheid van een gebeurtenis $A$ ten opzichte van gebeurtenis $B$ interpreteren als dat het waar zijn van gebeurtenis $B$
geen invloed heeft op de kans van $A$.

\begin{proposition}[Productregel]
    Laat $\Omega$ een kansruimte met kansverdeling $P$ en $A,B \subseteq \Omega$ twee deelverzamelingen.
    Er geldt $P(A \cap B) = P(A)P(B)$ precies dan als $A$ en $B$ onafhankelijk zijn.
\end{proposition}
\begin{proof}
    Merk op dat de volgende vergelijkingen equivalent zijn.
    \begin{align*}
        P(A|B)                      &= P(A) & \text{definitie onafhankelijkheid}\\
        \frac{P(A \cap B)}{P(B)}    &= P(A) \\
        P(A \cap B)                 &= P(A)P(B)
    \end{align*}
    Dit bewijst de gevraagde gelijkwaardigheid.
\end{proof}

\begin{proposition}[De regel van Bayes]\label{bayes}
    Laat $\Omega$ een kansruimte met kansverdeling $P$ en $A,B \subseteq \Omega$ twee deelverzamelingen.
    Dan geldt $P(A|B) = \frac{P(B|A)P(A)}{P(B)}$.
\end{proposition}
\begin{proof}
    Het bewijs is een opgave voor de lezer.
\end{proof}

\begin{exercise}
    Zij $\Omega = \set{1,\dots,20}$ de verzameling van uitkomsten van het gooien van een twintigzijdige dobbelsteen.
    Zij $A \subseteq \Omega$ de verzameling van oneven uitkomsten.
    Bereken de volgende uitdrukkingen.
    \begin{enumerate}[a.]
        \item $P(A)$
        \item $P(\set{5}|A)$
        \item $P(\set{1,\dots,10}|A)$
        \item $P(A|\set{4})$
        \item $P(A|\set{5})$
        \item $P(A|\set{1,\dots,7})$
    \end{enumerate}
\end{exercise}

\begin{exercise}
    Bewijs Propositie~\ref{bayes}.
\end{exercise}

\begin{exercise}
In een groep van 30 studenten, zeven hebben het huiswerk niet gemaakt. De docent kies willekeurig zes studenten.
Wat is de kans dat tenminste vier van hen het huiswerk niet hebben gemaakt?
\end{exercise}

\begin{exercise}
Er zitten 30 artikelen in een doos. Daarvan zijn er drie beschadigd.
5 willekeurige artikelen worden uit de doos gehaald. Wat is de kans dat hoogstens twee daarvan beschadigd zijn?
\end{exercise}

Het kan ook zo zijn dat twee deelverzamelingen onafhankelijk zijn als we ze beperken tot een zekere derde deelverzamelingen.
Of met andere woorden, dat twee gebeurtenissen onafhankelijk zijn gegeven een derde gebeurtenis.
Dit noemen we voorwaardelijke onafhankelijkheid.
\begin{definition}
    Laat $\Omega$ een kansruimte met kansverdeling $P$ en $A,B,C \subseteq \Omega$ drie deelverzamelingen.
    Dan heten $A$ en $B$ onafhankelijk gegeven $C$ als $P(A|B \cap C) = P(A|C)$.
\end{definition}

\begin{proposition}
    Laat $\Omega$ een kansruimte met kansverdeling $P$ en $A,B,C \subseteq \Omega$ drie deelverzamelingen.
    Er geldt dat $P(A \cap B|C) = P(A|C)P(B|C)$ precies dan als $A$ en $B$ onafhankelijk zijn gegeven $C$.
\end{proposition}
\begin{proof}
    Merk op dat de volgende vergelijkingen equivalent zijn.
    \begin{align*}
        P(A \cap B|C)                       &= P(A|C)P(B|C) \\
        \frac{P(A \cap B \cap C)}{P(C)}     &= \frac{P(A \cap C)}{P(C)} \frac{P(B \cap C)}{P(C)} \\
        P(A \cap B \cap C)P(C)              &= P(A \cap C)P(B \cap C) \\
        P(A \cap B \cap C)P(B \cap C)       &= P(A \cap C)P(C) \\
        P(A | B \cap C)                     &= P(A | C)
    \end{align*}
    Dit bewijst de gevraagde gelijkwaardigheid.
\end{proof}


\section{Product tussen kansruimtes}
Als we nu twee kansruimtes $A$ en $B$ hebben met elk hun eigen ware uitkomst, en we deze ruimtes tegelijk beschouwen, dan is er precies \'e\'en
combinatie van twee elementen uit respectievelijk $A$ en $B$ die de twee ware uitkomsten bevat.
Met andere woorden, als we ge\"{\i}nteresseerd zijn in beide kansruimtes tegelijk, dan kunnen we de gezamenlijke kansruimte goed karakteriseren
door middel van een cartesisch product.
Als de $A$ en $B$ bovendien onafhankelijk zijn, is de kans van elk paar uitkomsten gemakkelijk uit te drukken in termen van de kansverdelingen van $A$
en $B$.

\begin{definition}
    Laat $A,B$ twee onafhankelijke kansruimtes met verdelingen $P_A,P_B$.
    Dan vormt $A \times B$ een nieuwe kansruimte met verdeling $P_{A \times B}((a,b)) = P_A(a) \cdot P_B(b)$.
\end{definition}

\begin{exercise}
Wat is de kans om vier maal op een rij het getal 1 te gooien met een zeszijdige dobbelsteen?
\end{exercise}

\begin{exercise}
Student A, B en C willen willekeurig bepalen wie er bier gaat halen.
Ze komen overeen een munt op te gooien om te bepalen welk van de drie deze taak op zich zal nemen.
Ze hebben echter maar 1 munt met 2 zijden, terwijl ze met 3 personen zijn. Student C stelt het volgende voor. De munt moet herhaaldelijk worden opgegooid
totdat 1 van de volgende situaties zich voordoet: als er op enig moment tweemaal achtereenvolgens munt boven komt dan wint student A, als tweemaal
achtereenvolgens kop boven komt wint student B, en als er eenmaal kop boven komt en direct daarna munt boven komt dan wint student C. Doet student A er
verstandig aan akkoord te gaan met het voorstel van student C, gegeven dat student A liever geen bier haalt?
\end{exercise}


\section{Verzamelingen van kansruimtes}
We kunnen een aantal proposities generaliseren voor verzamelingen van kansruimtes.

\begin{proposition}[kettingregel]
    Laat $\Omega$ een kansruimte met kansverdeling $P$ en $(A_i), 0 \leq i \leq n$ een rij deelverzamelingen van $\Omega$.
    Dan geldt $P(A_0 \cap \dots \cap A_n) = P(A_0|A_1 \cap \dots \cap A_n) \dots P(A_{n-1}|A_n)P(A_n)$.
\end{proposition}
\begin{proof}
    Het bewijs is een opgave voor de lezer.
\end{proof}

\begin{proposition}[marginale kans]
    Laat $\Omega$ een kansruimte met kansverdeling $P$ en $A \subseteq \Omega$ een deelverzamelingen.
    Laat bovendien $(B_i), 0 \leq i \leq n$ een partitie van $\Omega$.
    Dan geldt $P(A) = \sum_{i=0}^n P(A \cap B_i)$.
\end{proposition}
\begin{proof}
    Het bewijs is een opgave voor de lezer.
\end{proof}




\chapter{Stochastische variabelen}


\section{Functies op kansruimtes}
Gegeven een kansruimte en een functie op die kansruimte, kan er een nieuwe kansruimte geconstrueerd worden.

\begin{definition}[morfisme]\label{morfisme}
    Laat $A$ een kansruimte met kansverdeling $P_A$, $B$ een verzameling en $\varphi : A \to B$ een surjectieve functie.
    De functie $\varphi$ heet een \emph{morfisme} van $A$ naar $B$.

    Als $\varphi$ bijectief is noemen we $\varphi$ ook wel een \emph{isomorfisme}.
    Als er een isomorfisme bestaat tussen twee kansruimtes $A$ en $B$, dan heten $A$ en $B$ isomorf.
\end{definition}

\begin{proposition}
    Er geldt dat $B$ uit Definitie~\ref{morfisme} via de morfisme $\varphi$ opnieuw een kansruimte is,
    met een kansverdeling $P_B(b) = P_A(\set{a \in A}{\varphi(a) = b})$.
\end{proposition}
\begin{proof}
    We moeten laten zien dat $P_B$ aan de definitie van een kansverdeling voldoet.

    \begin{enumerate}[i]
        \item Laat $b \in B$. Dan geldt $P_B(b) =  P_A(\set{a \in A}{\varphi(a) = b}) \geq 0$.
        \item Per definitie weten we dat
        \begin{align*}
            P_B(B) &= \sum_{b \in B} P_B(b) \\
                   &= \sum_{b \in B} P_A(\set{a \in A}{\varphi(a) = b}) \\
        \end{align*}
        Omdat $\varphi$ een surjectieve functie is weten we dat elk element uit $A$ precies eenmaal voorkomt in bovenstaande sommatie.
        Dus \[ \sum_{b \in B} P_A(\set{a \in A}{\varphi(a) = b}) = \sum_{a \in A} P_A(a) = P(A) = 1. \]
    \end{enumerate}
\end{proof}


\section{Stochastische variabelen}

Als de elementen van een kansruimte uit te drukken zijn in een getal, dan kunnen we bepaalde numerieke eigenschappen uitrekenen.
Een stochastische variabele op een kansruimte is een morfisme naar een deelverzameling van $\reals$.

\begin{definition}
    Laat $\Omega$ een kansruimte met kansverdeling $P$, en laat $S \subseteq \reals$ en $X : \Omega \to S$ een morfisme.
    Dan heet $X$ een \emph{stochastische variabele} op $\Omega$, of kortweg, \emph{stochast}.
\end{definition}

Omdat we de elementen van kansruimte via een stochast in getallen kunnen uitdrukken, kunnen we spreken van de `verwachte waarde' van een stochast.
Dit is het gemiddelde van alle elementen, gewogen met kun kans.

\begin{definition}
    Laat $X$ een stochast op $\Omega$.
    Definieer \[ \mu_X = \sum_{\omega \in \Omega} X(\omega) \cdot P(\omega). \]
    Dan heet $\mu_X$ de verwachte waarde van $X$.
\end{definition}

In andere literatuur wordt vaak het symbool $E$ gebruikt in plaats van $\mu$.
De verwachte waarde van $X$ wordt ook genoteerd met $\mu(X)$ of $\mu[X]$, afhankelijk van wat het meest leesbaar is.


Als de kansruimte van een stochast uit de context duidelijk is, wordt gemakshalve deze vaak weggelaten uit de notatie.
Men doet dan alsof de stochast een soort variabele die waarden kan aannemen volgens een bepaalde kansverdeling.
De stochast wordt dan in feite behandeld als de kansruimte die gevormd wordt door zijn codomein.
Het is echter van belang dat men zich altijd bewust is van de onderliggende morfisme.


We kunnen spreken over een standaard deviatie $sigma$ van een stochast. Dit is te interpreteren als de verwachte afwijking van de verwachte waarde.

\begin{definition}
    Laat $X$ een stochast op $\Omega$.
    Definieer
    \begin{align*}
        \sigma_X &= \mu( |X - \mu_X| ) \\
                  &= \sum_{\omega \in \Omega} |X(\omega) - \mu_X| \cdot P(\omega) \\
    \end{align*}
    Dan heet $\sigma_X$ de standaard deviatie van $X$.
\end{definition}

Omdat de standaard deviatie in de praktijk onhandige algebraische eigenschappen heeft, wordt in plaats van standaard deviatie vaak variantie
$\sigma^2$ gebruikt.

\begin{definition}
    Laat $X$ een stochast op $\Omega$.
    Definieer
    \begin{align*}
        \sigma^2_X &= \mu[ {(X - \mu_X)}^2 ] \\
                    &= \sum_{\omega \in \Omega} {(X(\omega) - \mu_X)}^2 \cdot P(\omega) \\
    \end{align*}
    Dan heet $\sigma^2_X$ de variantie van $X$.
\end{definition}

\begin{exercise}
    Laat $D = \set{1,2,3,4,5,6}$ de kansruimte van een zeszijdige dobbelsteen, met uniforme verdeling $P$.
    Stel we hebben twee dobbelstenen van dit type, en we zijn ge\"{\i}nteresseerd in het totaal aantal ogen na een worp met beide stenen.
    \begin{enumerate}[a.]
    \item Laat zien hoe we de bijbehorende stochast kunnen construeren uit $D$ via een cartesisch product en een morfisme.
    \item Wat is de verwachte waarde van deze stochast?
    \item Wat is de standaard afwijking van deze stochast?
    \item Wat is de verwachte waarde gegeven dat de eerste dobbelsteenworp oneven is?
    \end{enumerate}
\end{exercise}


\section{Samenhang tussen stochastische variabelen}
Als we meerdere stochastische variabelen hebben, zijn we vaak geinteresseerd in de samenhang tussen deze variabelen.
We willen dan graag weten hoeveel ze met elkaar te maken hebben, i.e.\ als de ene variabele klein is, of dat de kans be\"{\i}vloed dat de andere
variabele ook klein is. We introduceren een aantal maten die vaak gebruikt worden bij het beschrijven van samenhang.

\begin{definition}
    Laat $X,Y$ twee stochasten op $\Omega$.
    Definieer
    \begin{align*}
        \sigma_{XY} &= \mu[ (X - \mu_X)(Y - \mu_Y) ] \\
                    &= \sum_{\omega \in \Omega} (X(\omega) - \mu_X) \cdot (Y(\omega) - \mu_Y) \cdot P(\omega) \\
    \end{align*}
    Dan heet $\sigma-{XY}$ de covariantie tussen $X$ en $Y$.
\end{definition}

\begin{definition}
    Laat $X,Y$ twee stochasten op $\Omega$.
    Definieer
    \begin{align*}
        \rho_{XY} &= \frac{\sigma_{XY}}{\sigma_X \sigma_Y} \\
    \end{align*}
    Dan heet $\rho_{XY}$ de \emph{Pearson product-moment c\"oefficient} tussen $X$ en $Y$, of gewoon \emph{correlatie co\"oefficient}.
\end{definition}

\begin{exercise}
    Laat $D = \set{1,2,3,4,5,6}$ de kansruimte van een zeszijdige dobbelsteen, met uniforme verdeling $P$.
    Stel we hebben twee dobbelstenen van dit type. Laat $\Omega$ de kansruimte behorend bij het werpen van beide dobbelstenen.
    Laat $X$ de stochast van het aantal ogen van de eerste dobbelsteen.
    Laat $Y$ de stochast van het aantal ogen van de tweede dobbelsteen.
    Laat $Z$ de stochast van het totaal aantal ogen van beide dobbelstenen.
    \begin{enumerate}[a.]
    \item Wat is de covariantie tussen $X$ en $Y$?
    \item Wat is de correlatie co\"efficient tussen $X$ en $Y$?
    \item Wat is de covariantie tussen $X$ en $Z$?
    \item Wat is de correlatie co\"efficient tussen $X$ en $Z$?
    \end{enumerate}
\end{exercise}


\chapter{Statistiek}

\end{document}

